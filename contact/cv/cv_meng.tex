%%%%%%%%%%%%%%%%%%%%%%%%%%%%%%%%%%%%%%%%%
% Medium Length Graduate Curriculum Vitae
% LaTeX Template
% Version 1.1 (9/12/12)
%
% This template has been downloaded from:
% http://www.LaTeXTemplates.com
%
% Original author:
% Rensselaer Polytechnic Institute (http://www.rpi.edu/dept/arc/training/latex/resumes/)
%
% Important note:
% This template requires the res.cls file to be in the same directory as the
% .tex file. The res.cls file provides the resume style used for structuring the
% document.
%
%%%%%%%%%%%%%%%%%%%%%%%%%%%%%%%%%%%%%%%%%

%----------------------------------------------------------------------------------------
%	PACKAGES AND OTHER DOCUMENT CONFIGURATIONS
%----------------------------------------------------------------------------------------

\documentclass[margin, 10pt]{res} % Use the res.cls style, the font size can be changed to 11pt or 12pt here

\usepackage{helvet} % Default font is the helvetica postscript font
%\usepackage{newcent} % To change the default font to the new century schoolbook postscript font uncomment this line and comment the one above

\setlength{\textwidth}{5.1in} % Text width of the document

\begin{document}

%----------------------------------------------------------------------------------------
%	NAME AND ADDRESS SECTION
%----------------------------------------------------------------------------------------

\moveleft.5\hoffset\centerline{\Large\bf Meng Jiang} % Your name at the top

\moveleft\hoffset\vbox{\hrule width\resumewidth height 1pt} % \smallskip % Horizontal line after name; adjust line thickness by changing the '1pt'

%\moveleft.5\hoffset\centerline{123 Broadway} % Your address
%\moveleft.5\hoffset\centerline{City, State 12345}
%\moveleft.5\hoffset\centerline{(000) 111-1111 or (111) 111-1112}

%----------------------------------------------------------------------------------------

\begin{resume}

%----------------------------------------------------------------------------------------
%	CONTACT INFORMATION SECTION
%----------------------------------------------------------------------------------------

\section{Contact \\ Information}

{Office 2130, Siebel Center for Computer Science} \hfill {Mobile:} {(+1) (217) 418 6072} \\
{University of Illinois at Urbana-Champaign} \hfill {E-mail:} {mjiang89@gmail.com} \\
{201 N Goodwin Avenue, Urbana, IL 61801, USA} \hfill {http://www.meng-jiang.com}

%----------------------------------------------------------------------------------------
%	RESEARCH INTERESTS SECTION
%----------------------------------------------------------------------------------------

\section{Research \\ Interests}

Data-driven Behavior Modeling (Google Scholar: cited by 434, h-index = 8)
\begin{itemize}
	\item {\em Methodology}: Modeling content, social context, spatiotemporal context,
		multiple sources and intentions with advanced learning and mining approaches.
	\item {\em Applications}: Pattern discovery, summarization, prediction,
		recommendation and suspicious behavior detection.
\end{itemize}

%----------------------------------------------------------------------------------------
%	EDUCATION SECTION
%----------------------------------------------------------------------------------------

\section{Education}

{\bf University of Illinois at Urbana-Champaign, IL, USA} \hfill {August 2015 - Present}
\begin{itemize} \itemsep -2pt % Reduce space between items
	\item Postdoctoral Research Associate, Data Mining Research Group in the Department of Computer Science at College of Engineering
	\item Advisor: Professor Jiawei Han
\end{itemize}

{\bf Tsinghua University, Beijing, China} \hfill {August 2010 - July 2015}
\begin{itemize} \itemsep -2pt % Reduce space between items
\item PhD of Engineering, Computer Science and Technology
\item Advisor: Professor Shiqiang Yang
%\item GPA: 85/100 (3.13/4.0)
\item Graduated with {\em Excellent Doctoral Dissertation} (Top 3 among 109 CS PhDs)
	and {\em Excellent PhD Graduate, Beijing City}
\end{itemize}

{\bf Carnegie Mellon University, Pittsburgh, PA, USA} \hfill {August 2012 - May 2013}
\begin{itemize} \itemsep -2pt % Reduce space between items
\item Visiting Student, Computer Science
\item Advisor: Professor Christos Faloutsos
\end{itemize}

{\bf Tsinghua University, Beijing, China} \hfill {August 2006 - July 2010}
\begin{itemize} \itemsep -2pt % Reduce space between items
\item Bachelor of Engineering, Computer Science and Technology
\item GPA: 88.7/100 %(3.45/4.0)
	; {\em Excellent Undergraduate Student} Award
\item Graduated with {\em Excellent Thesis of Undergraduate} (Top 4 among 160)
\end{itemize}

%----------------------------------------------------------------------------------------
%	JOURNAL PAPERS SECTION
%----------------------------------------------------------------------------------------

\section{Journal \\ Papers}

[J1] Lu Liu, Feida Zhu, {\bf Meng Jiang}, Jiawei Han, Lifeng Sun and Shiqiang Yang. ``Mining Diversity on Social Media Networks'', in Multimedia Tools and Applications (MTA), 2012. (Regular paper, IF = {\em 1.35})

[J2] {\bf Meng Jiang}, Peng Cui, Fei Wang, Wenwu Zhu and Shiqiang Yang. ``Scalable Recommendation with Social Contextual Information'', in IEEE Transactions on Knowledge and Data Engineering (TKDE), 2014. (Regular paper, IF = {\em 2.07}, 25 citations till 06/2016)

[J3] {\bf Meng Jiang}, Peng Cui, Xumin Chen, Fei Wang, Wenwu Zhu and Shiqiang Yang. ``Social Recommendation with Cross-Domain Transferable Knowledge'', in IEEE Transactions on Knowledge and Data Engineering (TKDE), 2015. (Regular paper, IF = {\em 2.07})

[J4] {\bf Meng Jiang}, Peng Cui, Alex Beutel, Christos Faloutsos and Shiqiang Yang. ``Inferring Lockstep Behavior from Connectivity Pattern in Large Graphs'', in Knowledge and Information Systems (KAIS), 2015. (Regular paper, IF = {\em 1.78})

[J5] {\bf Meng Jiang}, Peng Cui and Christos Faloutsos. ``Suspicious Behavior Detection: Current Trends and Future Directions'', Special Issue on Online Behavioral Analysis and Modeling, in IEEE Intelligent Systems Magazine (ISSI), 2016. (Survey paper, IF = {\em 2.34})

[J6] {\bf Meng Jiang}, Peng Cui, Alex Beutel, Christos Faloutsos and Shiqiang Yang. ``Catching Synchronized Behaviors in Large Networks: A Graph Mining Approach'', Special Issue on the Best of SIGKDD 2014, in ACM Transactions on Knowledge Discovery from Data (TKDD), 2016. (To appear, regular paper, IF = {\em 0.93})

[J7] {\bf Meng Jiang}, Alex Beutel, Peng Cui, Bryan Hooi, Shiqiang Yang and Christos Faloutsos. ``Spotting Suspicious Behaviors in Multimodal Data: A General Metric and Algorithms'', in IEEE Transactions on Knowledge and Data Engineering (TKDE), 2016. (To appear, regular paper, IF = {\em 2.07})

%----------------------------------------------------------------------------------------
%	CONFERENCE PAPERS SECTION
%----------------------------------------------------------------------------------------

\section{Conference \\ Papers}

[C1] Lu Liu, Jie Tang, Jiawei Han, {\bf Meng Jiang} and Shiqiang Yang. ``Mining Topic-Level Influence in Heterogeneous Networks'', in the 19th ACM International Conference on Information and Knowledge Management (CIKM), 2010. (Full paper, 168 citations till 06/2016)

[C2] {\bf Meng Jiang}, Peng Cui, Rui Liu, Qiang Yang, Fei Wang, Wenwu Zhu and Shiqiang Yang. ``Social Contextual Recommendation'', in the 21st ACM International Conference on Information and Knowledge Management (CIKM), 2012. (Full paper, acceptance rate = {\em 13.4\%}, 104 citations till 06/2016)

[C3] {\bf Meng Jiang}, Peng Cui, Fei Wang, Qiang Yang, Wenwu Zhu and Shiqiang Yang. ``Social Recommendation across Multiple Relational Domains'', in the 21st ACM International Conference on Information and Knowledge Management (CIKM), 2012. (Full paper, acceptance rate = {\em 13.4\%}, 41 citations till 06/2016)

[C4] {\bf Meng Jiang}, Peng Cui, Alex Beutel, Christos Faloutsos and Shiqiang Yang. ``Detecting Suspicious Following Behavior in Multimillion-Node Social Networks'', in the 23rd International Conference on World Wide Web Companion (WWW), 2014. (Poster, 14 citations till 06/2016)

[C5] {\bf Meng Jiang}, Peng Cui, Alex Beutel, Christos Faloutsos and Shiqiang Yang. ``Inferring Strange Behavior from Connectivity Pattern in Social Networks'', in the 18th Pacific-Asia Conference on Knowledge Discovery and Data Mining (PAKDD), 2014. (Regular paper with long presentation, acceptance rate = {\em 10.8\%}, 22 citations till 06/2016)

[C6] {\bf Meng Jiang}, Peng Cui, Fei Wang, Xinran Xu, Wenwu Zhu and Shiqiang Yang. ``FEMA: Flexible Evolutionary Multi-faceted Analysis for Dynamic Behavioral Pattern Discovery'', in the 20th ACM SIGKDD International Conference on Knowledge Discovery and Data Mining (KDD), 2014. (Full paper, acceptance rate = {\em 14.6\%}, 10 citations till 06/2016)

[C7] {\bf Meng Jiang}, Peng Cui, Alex Beutel, Christos Faloutsos and Shiqiang Yang. ``CatchSync: Catching Synchronized Behavior in Large Directed Graphs'', in the 20th ACM SIGKDD International Conference on Knowledge Discovery and Data Mining (KDD), 2014. (Full paper, acceptance rate = {\em 14.6\%}, {\bf best paper finalist}: {\em top 9/151}, 27 citations till 06/2016)

[C8] {\bf Meng Jiang}, Alex Beutel, Peng Cui, Bryan Hooi, Shiqiang Yang and Christos Faloutsos. ``A General Suspiciousness Metric for Dense Blocks in Multimodal Data'', in the 15th IEEE International Conference on Data Mining (ICDM), 2015. (Short paper, acceptance rate = {\em 18.2\%}, 7 citations till 06/2016)

[C9] {\bf Meng Jiang}, Peng Cui, Nicholas Jing Yuan, Xing Xie and Shiqiang Yang. ``Little is Much: Bridging Cross-Platform Behaviors through Overlapped Crowds'', in the 30th AAAI Conference on Artificial Intelligence (AAAI), 2016. (Regular paper, acceptance rate = {\em 26\%})

[C10] {\bf Meng Jiang}, Christos Faloutsos and Jiawei Han. ``CatchTartan: Representing and Summarizing Dynamic Multicontextual Behaviors'', in the 22th ACM SIGKDD International Conference on Knowledge Discovery and Data Mining (KDD), 2016. (Full paper with oral presentation, acceptance rate = {\em 8.9\%})

%----------------------------------------------------------------------------------------
%	BOOK CHAPTER SECTION
%----------------------------------------------------------------------------------------

\section{Book chapter}

[B1] {\bf Meng Jiang} and Peng Cui. ``Mining User Behaviors in Large Social Networks'', in ``Big Data in Complex and Social Networks'', Chapman and Hall/CRC Big Data Series, 2016.

%----------------------------------------------------------------------------------------
%	TUTORIAL SECTION
%----------------------------------------------------------------------------------------

\section{Tutorial}

[T1] {\bf Meng Jiang} and Peng Cui. ``Behavior Modeling in Social Networks: From Micro to Macro'', in the 15th IEEE International Conference on Data Mining (ICDM), 2015.

%----------------------------------------------------------------------------------------
%	TALK SECTION
%----------------------------------------------------------------------------------------

\section{Invited Talk}

``Modeling complex behavior in social media'', at {\em University of Massachusetts Boston}, November 23, 2015. \\
%``My Ph.D. work in Tsinghua University'', at {\em Data Mining Research Group Meeting, UIUC}, September 18, 2015. \\
``Behavior prediction and anomaly detection in large-scale social networks'', at {\em Tencent Big Data Seminar}, July 25, 2014. \\
``User behavior analysis in social networks: pattern discovery, prediction and anomaly detection.'', at {\em MSRA}, July 8, 2014.
% ``Before data preprocessing'', at {\em Tsinghua University}, March 25, 2014.

%----------------------------------------------------------------------------------------
%	HONORS SECTION
%----------------------------------------------------------------------------------------

\section{Honors and Awards}

{Excellent PhD Graduate, Beijing City}, 2015 \\
{\bf Excellent Doctoral Dissertation, Tsinghua University}, 2015 \\
{National Scholarship, China}, 2014 \\
{Award of Excellence in Microsoft Research Asia Internship Program}, 2014 \\
{\bf KDD Best Paper Finalist}, 2014 \\
{KDD Student Travel Award}, 2014 \\
{PAKDD Student Travel Grant Award}, 2014 \\
{Sohu Graduate Research Fellowship}, 2013 \\
{Excellent Thesis of Undergraduate Students}, 2010 \\
{Excellent Undergraduate Student Award}, 2010 \\
{Tsinghua-Toyota Scholarship}, 2008

%----------------------------------------------------------------------------------------
%	RESEARCH EXPERIENCE SECTION
%----------------------------------------------------------------------------------------

\section{Research \\ Experience}

{\bf University of Illinois at Urbana-Champaign, College of Engineering, Department of Computer Science} \\
{\em Postdoctral Research Associate for Prof. Jiawei Han} \hfill {August 2015 - Present} \\
My research is constructing and mining heterogeneous behavioral networks.

{\bf Carnegie Mellon University, Computer Science Department} \\
{\em Visiting Student in Prof. Christos Faloutsos's Group} \hfill {August 2012 - May 2013} \\
My research has ranged from mining large graphs of ``who-follows-whom'' networks to detecting fraudulent behavior in social networks.

{\bf Tsinghua University, Department of Computer Science and Technology} \\
{\em Research Assistant for Prof. Shiqiang Yang} \hfill {August 2010 - July 2015} \\
{\em Teaching Assistant for Multimedia Computing} \hfill {Spring 2011} \\
My research is mining user behaviors, ranging from social recommendation to user behavior pattern discovery in temporal data.

%----------------------------------------------------------------------------------------
%	PROJECT EXPERIENCE SECTION
%----------------------------------------------------------------------------------------

\section{Project \& \\ Internship \\ Experience}

{\bf Tsinghua-Tencent APP Recommendation Project} \\
{\em Team Leader} (Mentor: Chuan Chen) \hfill {September 2014 - June 2015} \\
Developed APP downloading/installing behavior prediction algorithms by integrating cross-domain and social data.

{\bf Microsoft Research Asia (MSRA)} \\
{\em Full-time Research Intern} (Mentor: Dr. Xing Xie) \hfill {July 2014 - September 2014} \\
Research on constructing a unified network and transfer learning across social platforms (Douban and Weibo).
I got Award of Excellence in MSRA Intern Program.

{\bf Tsinghua-Tencent Joint Lab} \\
{\em Software Engineering Intern} (Mentor: Gordon Sun) \hfill {April 2011 - March 2012} \\
Research on improving Tencent search engine SOSO performance with social media data from Tencent Weibo.
% The system achieved 20.7\% increase in prediction accuracy.

{\bf Tsinghua-Samsung User Intention Inference Project} \\
{\em Team Leader} (Mentor: Dr. Yuanyuan Shi) \hfill {March 2011 - January 2012} \\
As the team leader, designed and implemented an Android application ``Come On'' that collects and recommends social activities in campus.
Developed a user intention inference kernel algorithm and wrote the concluding report.

{\bf Graphics Software Project} \\
{\em Research Assistant} (Mentor: Dr. Shimin Hu) \hfill {May 2009 - August 2009} \\
Developed a graphical plug-in in C\# project implementing topological repair algorithm of solid model using skeleton.

%----------------------------------------------------------------------------------------
%	INTERNSHIP EXPERIENCE SECTION
%----------------------------------------------------------------------------------------

% \section{Internship \\ Experience}

%----------------------------------------------------------------------------------------
%	SERVICE SECTION
%----------------------------------------------------------------------------------------

\section{Services}

{\bf Special session chair:} BBDA in IEEE DSAA 2016 \\
{\bf PC member:} IEEE DSAA 2015, DSC 2016 \\
{\bf Reviewer:}
Frontiers of Computer Science (FCS),
IEEE Intelligent Systems (IS),
INFORMS Journal on Computing (JOC),
Information Sciences (INS),
Journal of Computers (JCP),
Knowledge and Information Systems (KAIS),
Neurocomputing (NEWCOM),
Trans. on Information Forensics and Security (TIFS),
Trans. on Intelligent Systems and Technology (TIST),
Trans. on Internet and Information Systems (TIIS),
Trans. on Knowledge Discovery from Data (TKDD),
Trans. on Knowledge and Data Engineering (TKDE),
Trans. on Multimedia Computing Communications and Applications (TOMCCAP),
World Wide Web Journal (WWWJ) \\
{\bf External reviewer:} EDBT 2016, AAAI 2016, KDD 2015, IJCAI 2015, AAAI 2015,
WSDM 2015, ICDM 2014, CIKM 2014, KDD 2014, Globecom 2014 \\
{\bf Volunteer:} KDD 2014

%----------------------------------------------------------------------------------------
%	ACTIVITIES SECTION
%----------------------------------------------------------------------------------------

\section{Activities}

%Co-founder of ``Peerii'', a P2P Mobile Comm. Tech. Start-up \hfill {Jan. 2010 - Dec. 2011} \\
Career Guidance, Computer Science Department, Tsinghua \hfill {July 2010 - July 2011} \\
Vice President of Students' Union \hfill {May 2008 - May 2009}

%----------------------------------------------------------------------------------------
%	SKILLS SECTION
%----------------------------------------------------------------------------------------

\section{Skills}

Python, R, Matlab, JAVA, C++, C\# \\

\end{resume}
\end{document}
